%\usepackage[dvips]{graphicx}
\usepackage{epsfig}
%\usepackage[chapter]{theorems}
\usepackage{symbols}
\usepackage{url}
\usepackage{algorithm}
\usepackage{algorithmic}
\usepackage{color}
\usepackage{alltt}
\usepackage{dsfont}
\usepackage{upgreek}
\usepackage[tight]{shorttoc}
\usepackage{multirow}
\usepackage{rotating}
\usepackage{array}
\usepackage{colortbl}
\usepackage{xcolor}
\usepackage{listings}
\usepackage[rounded]{syntax}
\usepackage{slashbox}
\usepackage{graphicx, subfig}
\usepackage{enumerate}
\usepackage{amssymb}
\usepackage{appendix}
\usepackage{eurosym}
\usepackage{amsfonts}
\usepackage{pifont}
%\usepackage{hyperref}
\usepackage{pdflscape}
\usepackage{amsmath}
\usepackage{fixltx2e} 

\lstset{ %
        language=Prolog,                % the language of the code
        basicstyle=\scriptsize,       % the size of the fonts that are used for the code
        %numbers=right,                   % where to put the line-numbers
        %numberstyle=\scriptsize,      % the size of the fonts that are used for the line-numbers
        %stepnumber=1,                   % the step between two line-numbers. If it's 1, each line 
                                % will be numbered
        %numbersep=0.05cm,                  % how far the line-numbers are from the code
        backgroundcolor=\color{white},  % choose the background color. You must add \usepackage{color}
        showspaces=false,               % show spaces adding particular underscores
        showstringspaces=false,         % underline spaces within strings
        showtabs=false,                 % show tabs within strings adding particular underscores
        frame=none,                   % adds a frame around the code
        tabsize=3,                      % sets default tabsize to 2 spaces
        captionpos=b,                   % sets the caption-position to bottom
        breaklines=true,                % sets automatic line breaking
        breakatwhitespace=false,        % sets if automatic breaks should only happen at whitespace
        %title=\lstname,                 % show the filename of files included with \lstinputlisting;
                                   % also try caption instead of title
        escapeinside={\%*}{*)},         % if you want to add a comment within your code
        deletekeywords={not}
        %morekeywords={luents, actions, always, initial, if, inertial, after, inertial, noConcurrency,goal,not,caused,executable,nonexecutable,forbiden,requires}            % if you want to add more keywords to the set
}
\newcommand{\hlight}[2]{\begingroup \color{#1}#2 \endgroup}
\newcommand{\manolo}[2]{{\hlight{#1}{#2}}}

\def\ie{\textit{i.e.}}
\def\eg{\textit{e.g.}}
\def\cf{\textit{cf.}}
\def\aka{\textit{a.k.a.}}
