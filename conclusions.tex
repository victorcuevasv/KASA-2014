
\section{Conclusions and future work} \label{sec:conclusions}

In this paper we presented a language and system for the specification and enactment of data-centric workflows based on service composition. In addition, we introduced a planning-based approach for the enumeration of the search space of workflows implementing requirements specifications. Concretely, we proposed a set of constraints modeled in an action language, specifically DLV-K, in order to characterize the transformation of workflows with sequential and parallel compositions. This work is envisaged to be a foundation for incorporating a full cost model that covers the specification of composite computation services, leading to the selection of the most suitable workflow w.r.t. the user's preferences. Future work also includes validating the practicality of ASASEL for the specification of data-centric workflows for diverse users, which would require a more sophisticated GUI-based editing tool than our current prototype. 
