
\section{Related work} \label{sec:relatedwork}

The optimization of workflows involving data services must tackle challenges such as autonomy, non-durability of data, absence of fine grain data statistics, etc. For instance, the evaluation of queries over Web services presented in \cite{Srivastava2006} proposes an optimization approach by ordering the service calls in a pipelined fashion and by tunning the size of data. However, the control over data size (i.e. data chunks) and selectivity statistics is a strong assumption because of the autonomy of services and the absence of data statistics in many real scenarios.

Another aspect to consider during the optimization is the selection of the service instances, which represents a factor for the service coordination cost. In \cite{Claro2005,Wada2008} the optimal selection of services with a multidimensional cost is proposed for optimizing the complete service coordination. This selection is done by solving a multi-objective assignation problem for a set of abstract services. However, the possibility to modify the order of service invocations is not considered and this is a key issue in service-oriented workflow optimization.

