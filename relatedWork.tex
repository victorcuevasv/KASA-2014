
\section{Related work} \label{sec:relatedwork}

At their top level, service-oriented workflows involving data operations share some similarities with queries over Web services as presented in \cite{Srivastava:2006:QOO:1182635.1164159}. There the authors propose an optimization approach by ordering the service calls in a pipelined fashion and by tuning the size of service call batches. An algebraic approach for the optimization of workflows with relational and map-reduce operations is presented in \cite{DBLP:journals/pvldb/OgasawaraOVDPM11}. Our approach is to enable workflows with a broader variety of operations defined through service compositions, thus requiring alternative optimization techniques.

Alternative formalisms for the specification of workflows include, for example, process algebras \cite{Curcin:2011:STW:2048456.2048467} and petri nets \cite{Hidders:2008:DDL:1340791.1340907}. Although ASMs have been used to study and model the properties of workflows, less effort has been given to using them in a fully operational manner. The combination of data-flow and control-flow present in our approach has some similarities with the work presented in \cite{Bowers:2006:ESR:1129755.1130113}. However, they focus on complementing a workflow model with control-flow structures for tasks such as fault-tolerant and adaptive distributed data transfer, whereas we focus on enabling the composition of services to support new functionality.

