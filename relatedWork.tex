
\section{Related work} \label{sec:relatedWork}

Data-centric workflows involving services share some similarities with queries over Web services as presented in \cite{Srivastava:2006:QOO:1182635.1164159}. There the authors propose an optimization approach by ordering the service calls in a pipelined fashion and by tuning the size of service call batches. An algebraic approach for the optimization of workflows with relational and map-reduce operations is presented in \cite{DBLP:journals/pvldb/OgasawaraOVDPM11}. Our approach is to enable workflows with a broader variety of operations defined through service compositions, thus requiring alternative optimization techniques.

Planning techniques have been applied for automatic service composition, for instance in \cite{DBLP:conf/kr/McIlraithS02} and \cite{Sirin:2004:HPW:1741306.1741331}. The problem addressed in those works is to create a service composition from atomic actions (services) based on a propositional goal. The Roman Model \cite{DBLP:journals/debu/CalvaneseGLMP08} alternatively employs finite state transition system descriptions for the available and target services, but with the same basic objective in mind. However, we use planning techniques instead for the optimization of a workflow that includes possibly composite computation services.

Alternative formalisms for the specification of workflows include, for example, process algebras \cite{Curcin:2011:STW:2048456.2048467} and petri nets \cite{Hidders:2008:DDL:1340791.1340907}. The use of ASMs provides a formal semantics, as in the aformentioned formalisms, but also fully compatible text and workflow representations that are easy to specify. Although ASMs have been used to study and model the properties of workflows, less effort has been given to using them in a fully operational manner.

