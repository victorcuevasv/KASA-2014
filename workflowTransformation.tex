
\lstset{ %
        language=Prolog,                % the language of the code
        basicstyle=\scriptsize,       % the size of the fonts that are used for the code
        %numbers=right,                   % where to put the line-numbers
        %numberstyle=\scriptsize,      % the size of the fonts that are used for the line-numbers
        %stepnumber=1,                   % the step between two line-numbers. If it's 1, each line 
                                % will be numbered
        %numbersep=0.05cm,                  % how far the line-numbers are from the code
        backgroundcolor=\color{white},  % choose the background color. You must add \usepackage{color}
        showspaces=false,               % show spaces adding particular underscores
        showstringspaces=false,         % underline spaces within strings
        showtabs=false,                 % show tabs within strings adding particular underscores
        frame=none,                   % adds a frame around the code
        tabsize=3,                      % sets default tabsize to 2 spaces
        captionpos=b,                   % sets the caption-position to bottom
        breaklines=true,                % sets automatic line breaking
        breakatwhitespace=false,        % sets if automatic breaks should only happen at whitespace
        %title=\lstname,                 % show the filename of files included with \lstinputlisting;
                                   % also try caption instead of title
        escapeinside={\%*}{*)},         % if you want to add a comment within your code
        deletekeywords={not},
        aboveskip=5pt,
        belowskip=-3pt,
        %morekeywords={luents, actions, always, initial, if, inertial, after, inertial, noConcurrency,goal,not,caused,executable,nonexecutable,forbiden,requires}            % if you want to add more keywords to the set
}

\section{Workflow enumeration} \label{sec:workflowTransformation}

This section decribes the process of enumerating all the equivalent workflows that satisfy the same functional requirements given by an ASASEL specification. The enumeration leads to a search space of workflows with increasing levels of parallelism in their structure. The levels of parallelism can privilege the cost preferences such as response time or the communication cost. The enumeration is subject to constraints for composing the required activities by the ASASEL specification. We model these constraints as action rules in the language DLV-K\footnote{http://www.dbai.tuwien.ac.at/proj/dlv/k}.

In DLV-K, planning problems have a set of facts that represent the problem domain named background knowledge. The facts are predicates of static knowledge and are the input of the planning problem. Planning problems are modeled as state machines described by a set of fluents and a set of actions. A fluent is a property of an object in the world and is part of the states of the world. Fluents may be true, false or unknown. An action is executable if a precondition holds in the current state. Once an action is executed, the fluents and thus the state of the plan are modified. The action rules define the subset of fluents that must be held before the execution of an action (\ie{} pre-conditions) and the subset of fluents to be held after the execution (\ie{} post-conditions). Finally, a goal is a set of fluents that must be reached at the end of the plan. A goal is expressed by the conjunction of fluents and by a plan length $l \in \mathds{Z}^{+}$.

The mapping from workflow enumeration to a planning problem is shown in Table \ref{tab:mappingQW-PP}. The APIs and the required activities by the ASASEL specification are modeled as facts of the background knowledge. The execution state of a workflow is modeled as fluents and the activities to perform as actions.
	
	\begin{table}
	   \begin{center}
	      \begin{tabular}{|r|l|}
	         \hline
	         \multicolumn{1}{|c|}{\textbf{Workflow }}& \multicolumn{1}{c|}{\textbf{Planning problem}} \\
	         \hline
	         APIs, required activities & Facts (background knowledge) \\
	         \hline
	         Workflow states & Fluents \\
	         \hline
	         Workflow activities & Actions \\
	         \hline
	         Result delivery & Goal: \texttt{finished?(}$l \in \mathds{Z}^{+}$\texttt{)} \\
	         \hline
	      \end{tabular}
	   \end{center} 
	   \caption{Mapping to a planning problem}
	   \label{tab:mappingQW-PP}
	\end{table}
	
Next we show, through an example, how we represent the background knowledge for workflow enumeration. Afterwards, we show how the workflow state and activities are expressed in DLV-K rules. Given such rules, the DLV-K engine performs the workflow enumeration.

\subsection{Background knowledge} \label{subsec:kb}
The background knowledge contains a set of facts that serve as the input for the workflow enumeration. It includes (1) the service methods and (2) the required activities derived from the ASASEL specification.
               
\textbf{Service methods} are represented by the facts \texttt{method/2}. The bound and free attributes associated to such a method are represented by the facts \texttt{bound_p/4} and \texttt{free_p/4}. The rule \texttt{att/4} represents the normal form of an attribute.

%\lstinputlisting[lastline=15, fontadjust=true, label=list:service-interfaces,caption={Service interfaces}]{code/service-interfaces-FF.kb}

\lstinputlisting[fontadjust=true, label=list:service-interfaces]{code/service-interfaces-FF.kb}
\vspace*{0.25cm}
\textbf{Required activities} are derived from the ASASEL workflow specification and represented through facts (with the underscore at the end). The required activities derived from a workflow implementing ``\textit{What are the interests of my friend Joe}'' are represented by the following facts. 

%\lstinputlisting[fontadjust=true, label=list:hybridQuery, caption={Hybrid query in facts}]{code/query-FF1.kb}

\lstinputlisting[fontadjust=true, label=list:hybridQuery]{code/query-FF1.kb}
\vspace*{0.25cm}
These required activities express the need over the methods \texttt{p:profl} and \texttt{i:interests}. Both data are retrieved by \texttt{retrive\_/3} and represented by \texttt{p1} and \texttt{i1}. The nickname attribute of the profile is filtered by \texttt{filter\_/2} and correlated by \texttt{join\_/4} interests through the nickname attribute. The attributes nickname, score and tag are projected. Observe that the filter over the nickname attribute is only indicated as the equality operators are not relevant for the workflow transformation.

\subsection{Workflow activities}
Workflow activities are represented as actions in DLV-K. Such actions are predicates that require facts from the background knowledge to be true. There are also activities that are independent from facts.
         
\textbf{init and finish} These activities have the special purpose to initialize and terminate the workflow execution. Thus their semantics is not associated with the application and there is no dependency with the background knowledge.

\textbf{data\_service} establishes a connection with a data service method. It requires from the knowledge base a service method and the expressed need to retrieve data from it.

%\lstinputlisting[firstline=23, lastline=23, fontadjust=false, label=list:qw-on-demand, caption={on-demand activity}]{code/aesop.plan}

\lstinputlisting[firstline=23, lastline=24, fontadjust=false, label=list:qw-on-demand]{code/aesop2.plan}
\vspace*{0.25cm}
\textbf{bind\_selection} invokes a service method and retrieves data from it. The invocation is done by providing a bound attribute. 

%\lstinputlisting[firstline=26, lastline=27, fontadjust=false, label=list:qw-bind-selection, caption={bind-selection activity}]{code/aesop.plan}

\lstinputlisting[firstline=27, lastline=28, fontadjust=false, label=list:qw-bind-selection]{code/aesop2.plan}
\vspace*{0.25cm}
\textbf{bind_join} correlates data from two service methods w.r.t. an attribute from each one. The attribute from the outer method must be bound. This activity is analogous to \texttt{bind_selection} but it takes the value from another method attribute.
                     
%\lstinputlisting[firstline=30, lastline=35, fontadjust=false, label=list:qw-bind-join, caption={bind-join activity}]{code/aesop.plan}

\lstinputlisting[firstline=31, lastline=35, fontadjust=false, label=list:qw-bind-join]{code/aesop2.plan}
\vspace*{0.25cm}           
\textbf{filter} performs the filtering over an attribute of a required service method.
            
%\lstinputlisting[firstline=39, lastline=42, fontadjust=false, label=list:qw-select, caption={select activity}]{code/aesop.plan}

\lstinputlisting[firstline=39, lastline=42, fontadjust=false, label=list:qw-select]{code/aesop.plan}
\vspace*{0.25cm}                
\textbf{project} projects an attribute of a service method.
                     
%\lstinputlisting[firstline=44, lastline=44, fontadjust=false, label=list:qw-project, caption={project activity}]{code/aesop.plan}

\lstinputlisting[firstline=44, lastline=44, fontadjust=false, label=list:qw-project]{code/aesop.plan}
\vspace*{0.25cm}			   
The semantics of these activities is completed with constraints that define their pre-conditions and post-conditions.
				
\subsection{Workflow constraints} 
The workflow constraints define the pre-conditions and post-conditions associated to the execution of the workflow activities. A condition is a state of knowledge modifiable by the execution of activities. Through the satisfaction of such conditions, the workflows are transformed.
In the following, we present the intuition of these constraints along with their rules in DLV-k.

               
\textbf{init and finish} The \textbf{init} activity has no previous activity and its pre-condition is that the workflow has not been \texttt{initiated}. As post-condition, it produces the state \texttt{initiated}. The last activity is \texttt{finish} and there is no other activity to be executed afterwards. Its pre-condition is that there is not evidence that the workflow is \texttt{finished} and the result has been \texttt{delivered} (See \texttt{output} activity below for details about \texttt{delivered}). The post-condition of \texttt{finish} is \texttt{finished} and this is the goal to be reached for the workflow transformation.
                            
%\lstinputlisting[firstline=49, lastline=52, fontadjust=false, label=list:init-activity, caption={\texttt{init} and \texttt{finish} activities}]{code/aesop.plan}

\lstinputlisting[firstline=49, lastline=52, fontadjust=false, label=list:init-activity]{code/aesop.plan}
\vspace*{0.25cm}        
\textbf{data\_service} Once initiated the workflow, the data services must be \texttt{connected(DS)}. This fluent is produced by the execution of the \texttt{data\_service(DS)} activity. 

%\lstinputlisting[firstline=55, lastline=56, fontadjust=false, label=list:on-demand-activity, caption={on-demand activity}]{code/aesop.plan}

\lstinputlisting[firstline=55, lastline=56, fontadjust=false, label=list:on-demand-activity]{code/aesop.plan}
\vspace*{0.25cm}
In order to retrieve all the required data, all data services should be connected. The fluent \texttt{all_connected} that is false if there is not evidence that a data service is connected. Otherwise, it is true.

%\lstinputlisting[firstline=58, lastline=59, fontadjust=false, label=list:all-connected-fluent, caption={all\_connected fluent}]{code/aesop.plan}

\lstinputlisting[firstline=58, lastline=59, fontadjust=false, label=list:all-connected-fluent]{code/aesop.plan}
\vspace*{0.25cm}
\textbf{bind_selection} It is only executable if there is not evidence that data from the data service \texttt{DS} have been retrieved and if there is a connection with \texttt{DS}. Once the bind selection is executed, the fluent \texttt{retrieved(DS)} is true. 

      
%\lstinputlisting[firstline=61, lastline=63, fontadjust=false, label=list:bind-selection-activity, caption={bind\_ selection activity}]{code/aesop.plan}

\lstinputlisting[firstline=61, lastline=63, fontadjust=false, label=list:bind-selection-activity]{code/aesop.plan}
\vspace*{0.25cm}
\textbf{filter} It is executable if there is not evidence that the attribute \texttt{P} of \texttt{DS} has been filtered. It is required that the data from \texttt{DS} have been retrieved and the activity \texttt{select_(DS,P)} must be required. The execution of the filter makes the fluent \texttt{filtered(DS,P)} true.

%\lstinputlisting[firstline=74, lastline=76, fontadjust=false, label=list:select-activity, caption={select activity}]{code/aesop.plan}

\lstinputlisting[firstline=74, lastline=76, fontadjust=false, label=list:select-activity]{code/aesop2.plan}
\vspace*{0.25cm}
As might several filter activities over \texttt{DS} are required, the \texttt{all_filtered_from} becomes true if there is no other attribute pending to be filtere.

%\lstinputlisting[firstline=77, lastline=78, fontadjust=false, label=list:all-selected-from, caption={all\_select\_from fluent}]{code/aesop.plan}

\lstinputlisting[firstline=77, lastline=80, fontadjust=false, label=list:all-selected-from]{code/aesop2.plan}
\vspace*{0.25cm}
There is the fluent \texttt{all\_filtered} that becomes true if there is no other attribute of the method \texttt{DS} pending to be filtered.

%\lstinputlisting[firstline=80, lastline=82, fontadjust=false, label=list:all-selected, caption={all\_select fluent}]{code/aesop.plan}

\lstinputlisting[firstline=80, lastline=82, fontadjust=false, label=list:all-selected]{code/aesop.plan}
\vspace*{0.25cm}
\textbf{project} This activity is executable if there is not evidence that the attribute \texttt{P} of \texttt{DS} has been projected. The execution of projection makes the fluent \texttt{projected} true.
               
%\lstinputlisting[firstline=84, lastline=85, fontadjust=false, label=list:project, caption={project activity}]{code/aesop.plan}

\lstinputlisting[firstline=84, lastline=85, fontadjust=false, label=list:project]{code/aesop.plan}
\vspace*{0.25cm}
During the workflow execution, all the projection activities have to  be performed. For the method \texttt{DS}, the fluent \texttt{all\_projected\_from} is true if there is no other attribute from \texttt{DS} pending to be projected. The fluent \texttt{all\_projected} is true if there is no other \texttt{DS} with an attribute pending to be projected.
                
%\lstinputlisting[firstline=88, lastline=92, fontadjust=false, label=list:all-projected-from, caption={all\_projected\_from fluent}]{code/aesop.plan}

\lstinputlisting[firstline=88, lastline=94, fontadjust=false, label=list:all-projected-from]{code/aesop.plan}
\vspace*{0.25cm}                  
\textbf{output} Once all the required activities are performed, the result is delivered by the activity \texttt{output}. To model this pre-condition, the fluent \texttt{activities\_performed} is true if all the required activities have been processed. Otherwise, the fluent is false \texttt{-activities\_performed}.

%\lstinputlisting[firstline=95, lastline=99, fontadjust=false, label=list:query-processed, caption={query\_processed fluent}]{code/aesop.plan}

\lstinputlisting[firstline=95, lastline=99, fontadjust=false, label=list:query-processed]{code/aesop.plan}
\vspace*{0.25cm}
Once the result is delivered by \texttt{output}, the fluent \texttt{delivered} becomes true and the workflow can be finished (\cf{} \texttt{finish} pre-conditions).

%\lstinputlisting[firstline=101, lastline=102, fontadjust=false, label=list:output, caption={output fluent}]{code/aesop.plan}

\lstinputlisting[firstline=101, lastline=103, fontadjust=false, label=list:output]{code/aesop.plan}




